\chapter{Experimental Procedure}
\label{chp:DEM}
\section{Overview}
This chapter will discuss the experimental set-ups used to test the functionality and performance of the developed device. The aim of the study is to determine if the developed device can indeed measure accurate vital sign data from the ear canal. This, in turn, will answer the research question. Each measured vital sign needs to be validated, in order to prove that is is indeed accurate data.

Two types of validation will be used in this study: Benchmark validation for core temperature, heart rate, respiratory rate and Sp02; and event related potential detection to validate EEG.

Healthy adult volunteers will partake in this study. These volunteers will be fitted with the developed device and with the industry standard medical device. Device and benchmark data will be collected simultaneously and compared afterwards.

This study will test the actual data measured and also the processing of this data. For example the extraction of heart rate from PPG and the extraction of breathing rate from heart rate.

Tests will involve comparing time varying signals (PPG), time invariant signals and calculated figures (Breathing rate, Sp02, Temp?)

\section{Theory}
\section{Ethical Consent}
\section{Subjects}

\section{Benchmark Validation}
Core temperature, heart rate, respiratory rate and Sp02 measurements will be tested through benchmark validation. This entails comparing the measurements made by the developed device to measurements made, in the same conditions, by a industry standard medical device. In this study, a device that conforms to the EC requirements is seen as industry standard device. This is a valid assumption, for the CE mark is sign that the device complies with the EU legislation that is applicable to the product (what does this mean?).

Three devices was selected to provide the benchmark measurements. 


The vital sign measurements of the device will be compared to selected benchmarks. These benchmarks will be measurements made by various industry standard medical devices.





\section{Method}
\subsection{Benchmark Apparatus}
Benchmark devises are chosen to measure the same physiological signs as the developed device. The Nexus-10 physiological monitoring platform will be used to provide the benchmark measurements for PPG, heart rate and respiratory rate. The SureSense blablabla wil be used for the Sp02 benchmark and an ear thermometer for the core temperature benchmark.

Mind Media's Nexus-10 is a ten channel biofeedback system. It comes with a array of sensors that can acquire a range of different bio-signals. In this study the blood volume pulse, and respiration sensor will be used. The device can collect data at 128 samples per second. 

SureSense blablabla is a ...

Ear thermometer...

with photoplethysmograph was used. Photoplethysmograms was compared and average heart rate readings as well. This will evaluate the feasibility of measuring a PPG from the ear canal ass well as the extracting a heart rate from this signal.

\subsection{Comparing Data}
Comparing the device PPG to the Mexus BVP


\section{Results}


