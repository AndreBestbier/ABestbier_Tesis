\chapter{Concept Design}
\label{chp:ConDes}
This chapter builds upon the knowledge gained in the literature review and explains the logic used to select the methods and sensors to realize each medical sign monitoring requirement of the Ear-Monitor. Selections will be made by analysing advantages and disadvantages of each option and combining it with sound engineering judgement.

\section{Temperature}
The Ear-Monitor will measure core temperature from the inside of the ear canal. The main limitations are sensor size and measurement accuracy. The method selection and sensor selection is discussed separately.

\subsection{Temperature measurement method}
Two temperature measuring method were considered, namely contact- and non-contact thermometers.

\subsubsection{Contact thermometers}
A contact RTD, thermocouple or thermistor is placed in contact with the canal wall, canal air or tympanic membrane.

\begin{table}[H]
\caption{Contact thermometers evaluation}
\label{tab:ContactThermometersEval}
\renewcommand{\arraystretch}{1.3}	%Wat doen hierdie?
\centering
\begin{tabular}{ |Q{6cm}|Q{8cm}| } 
 \hline
 Image 									& Advantages  \\ 
 										& \tabitem Available in small sizes, ideal for the size restrictions of the ear canal.\\
 										& \tabitem Good accuracy.\\
 										& \tabitem Converting transduces voltage to temperature is simpler than with non-contact 		thermometers.\\
 \hline
 Sensor size: 0.5 x 2.3 mm 				& Disadvantages  \\ 
 Measurement accuracy: 0.15 $^{\circ}C$ & \tabitem Canal wall and canal air temperature measurements can easily be influenced by ambient temperature.\\
 										& \tabitem Tympanic membrane contact can cause discomfort and harm to the wearer.\\
 										& \tabitem More time is needed to take measurement, for sensor needs to be in thermal equilibrium with the object.\\
 
 \hline
\end{tabular}
\end{table}

\subsubsection{Non-contact Thermometers}
A non-contact, infrared sensor is placed inside the ear canal and pointed at the tympanic membrane.

