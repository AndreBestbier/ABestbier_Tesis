\chapter{Experimental Procedure}
\label{chp:DEM}
\section{Overview}
The Ear-Monitor is tested on a group of 16 healthy adult volunteers to calibrate sensors and verify the accuracy of its measurements. Comparative analysis is done on each of the medical signs measurements made by the Ear-Monitor by comparing its measurements to those made by selected clinical devices. Each participant is acting as his/her own control. Ethical clearance was obtained for these tests form HREC with reference number M16/09/038 (proof included in Appendix X).

\medskip



\section{Theory}
\section{Ethical Consent}
\section{Subjects}

\section{Benchmark Validation}
Core temperature, heart rate, respiratory rate and Sp02 measurements will be tested through benchmark validation. This entails comparing the measurements made by the developed device to measurements made, in the same conditions, by a industry standard medical device. In this study, a device that conforms to the EC requirements is seen as industry standard device. This is a valid assumption, for the CE mark is sign that the device complies with the EU legislation that is applicable to the product (what does this mean?).

Three devices was selected to provide the benchmark measurements. 


The vital sign measurements of the device will be compared to selected benchmarks. These benchmarks will be measurements made by various industry standard medical devices.





\section{Method}
\subsection{Benchmark Apparatus}
Benchmark devises are chosen to measure the same physiological signs as the developed device. The Nexus-10 physiological monitoring platform will be used to provide the benchmark measurements for PPG, heart rate and respiratory rate. The SureSense blablabla wil be used for the Sp02 benchmark and an ear thermometer for the core temperature benchmark.

Mind Media's Nexus-10 is a ten channel biofeedback system. It comes with a array of sensors that can acquire a range of different bio-signals. In this study the blood volume pulse, and respiration sensor will be used. The device can collect data at 128 samples per second. 

SureSense blablabla is a ...

Ear thermometer...

with photoplethysmograph was used. Photoplethysmograms was compared and average heart rate readings as well. This will evaluate the feasibility of measuring a PPG from the ear canal ass well as the extracting a heart rate from this signal.

\subsection{Comparing Data}
Comparing the device PPG to the Mexus BVP


\section{Results}


