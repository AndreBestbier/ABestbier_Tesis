\chapter{Introduction}
\label{chp:Intro}
This thesis document reports on a project undertaken in die biomedical field of wearable electronics. Great advances in miniaturization of electronics and wireless communication have challenged and transformed the norm of the how we use electronics to listen to the language of our bodies: bio-signals.

The importance and usefulness of a continuous, wearable health monitor should not be underestimated. Access to accurate, long term data can lead to improved diagnosis of health issues and a better understanding of how our bodies react to drugs, exercise, emotions and the environment around us.

Traditionally, bio-signals monitoring is done with stationary equipment and with a dedicated device for each signal to be measured. It is easy to see that this is not suitable for continuous and mobile bio-signals monitoring. 

This project concerns the design, development and testing of a proof of concept device that will overcome the limitations of these traditional methods. The device is to be worn on the ear and will transmit its collected data through a wireless connection to a supporting system for storage and analysis.


\section{Aim/Research Question}
To develop and test a proof of concept of a wearable device that can monitor bio-signals and transmit collected data wirelessly to a warning and storage system. Bio-signals include core temperature, heart rate, respiratory rate, blood oxygen saturation and electrical brain activity.
Is the external ear canal a feasible location for the continuous monitoring of core temperature, heart rate, respiratory rate, blood oxygen saturation and electrical brain activity by means of a ear worn device?

In order to achieve the aim of of this project the following three objectives have to be met:
\begin{itemize}
\item Develop an ear worn device to measure core temperature, heart rate, respiratory rate, blood oxygen saturation and electrical brain activity through the external ear.
\item Conduct a trail to determine the functionality of this device.
\item Subsequently, evaluate the feasibility of an ear worm bio-signal monitor
\end{itemize}

\section{Motivation}