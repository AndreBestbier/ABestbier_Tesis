\chapter{Introduction}
\label{chp:Introduction}
This thesis reports on a project undertaken in die biomedical field of wearable electronics. Great advances in the miniaturization of electronics and wireless communication have challenged and transformed the norm of how we use electronics to listen to the language of our bodies.

\medskip
This project revolves around the continuous measurement of vital signs. These signs are objective parameters that give an indication of physical well-being and the state of essential physiological functions. For example, infections are indicated by a rise in core temperature \citep{FluOrCold}, pneumonia can be detected by a shortness of breath \citep{MayoPneumonia}, an abnormal decrease in blood oxygen saturation during sleep can be a warning sign for sudden infant death syndrome \citep{thach2008tragic} and a rise in heart rate can indicate physical stress \citep{StressAndHR}. These signs can be detected electronically before traditionally observable symptoms appear. In many cases, the deciding factor in the success of a treatment is whether the illness is detected early enough.

\medskip
Because of this, the importance and usefulness of a continuous, wearable health monitor should not be underestimated. Access to accurate, long term data can lead to improved diagnosis of health issues and a better understanding of how our bodies react to drugs, exercise, emotions and the environment around us. Traditionally, vital sign monitoring is done with a stationary, dedicated device for each signal to be measured. Due to the many large, stationary equipment needed for traditional patient monitoring, it is obvious that this is not suitable for continuous and mobile vital sign monitoring.

\medskip
This project concerns the design, development and evaluation of a proof of concept device that will overcome the limitations of these traditional methods. The device is to be worn in the ear like an earphone or hearing aid. It will make multi-parameter vital sign measurements and transmit collected data through a wireless connection to a supporting system for storage and analysis. In this project, the supporting system is on a laptop, but it can also be on the smart-phone of the wearer or on a cloud server. This supporting system can be used by a physician, caretaker or the wearer self, to monitor and track his/her health.

\medskip
From here onwards, this device is referred to as the \textit{Ear-Monitor}. This report will discuss the project aim and objectives, relevant literature and the design, manufacturing and testing of the Ear-Monitor.

\section{Aim and Objectives}
To develop and test a proof of concept wearable device that can monitor vital signs through the external ear and transmit collected data wirelessly to a storage system. Vital signs include core temperature, heart rate, respiratory rate and blood oxygen saturation.

\medskip

In order to achieve the aim of this project the following three objectives have to be met:
\begin{itemize}
\item Develop a device to measure core temperature, heart rate, respiratory rate and blood oxygen saturation through the external ear of the wearer.
\item Conduct a trial experiment to test the device on a sample of human participants.
\item Use the data collected from the trial to evaluate the accuracy of the measurements made by the device.
\end{itemize}

\section{Motivation}
This project originated from a need found in medical practice and expressed by the proposer/advocate of this topic. It is the need for better vital sign monitoring methods for neonates and infants in hospital nurseries and at home. High-risk patients are placed in ICUs and are thoroughly monitored, whereas lower risk patients are left in the nursery or sent home \citep{barfield2012american}. These patients are poorly monitored while at a fragile age, increasing the risk of health issues. Insufficient health monitoring for neonates and infants is due to the lack of a practical monitoring method. The solution to this issue is the development of an unobtrusive, wearable health monitor.

\medskip

While contemplating and researching this idea, it was found that a much larger group can benefit from such a device. This lead to the project pivoting toward a more general purpose vital sign monitoring device. This device will prove if it is practical to measure the mentioned vital signs through the external ear canal. If this proof of concept is successful, the methods developed during this project can be used to develop specialized ear-worn devices for various applications. In practice, such a device can transmit health statistics and warnings in real time to a physician or caretaker. Applications include:

\begin{itemize}
\item Monitoring neonate- and infant health in nurseries and at home.
\item Monitoring health of patients with chronic illnesses.
\item Studying the effect of prescription drugs or other treatments.
\item Monitoring the health of people working under strenuous conditions like heavy machinery operators and soldiers.
\item Tracking the health and fitness of athletes.
\end{itemize}

The ear was chosen as location for various reasons. Firstly, the anatomy of the ear and the proximity of an ear-worn device to the tympanic membrane, means that al the mentioned vital signs mentioned can theoretically be measured from this location. This eliminates the need for multiple devices or the need for wires connecting sensors on different parts of the body. The absence of sensors on traditional locations such as the chest or limbs and the absence of connective wires mean that the ear-worn device is minimally obstructive for the wearer, especially through freeing up the hands and allowing free movement. Secondly, the shape of the external ear is ideal for supporting a device without the need for straps or adhesives. Furthermore, the head remains relatively still in relation to the rest of the body. This reduces the risk of motion artefacts corrupting the signals of interest. An ear-worn device can be embedded in the already familiar shape of an earphone or hearing aid. The final motivation for using the ear as location for the health monitor is its novelty. As will be apparent from Chapter \ref{chp:Literature Review} of this document, there is opportunity for research to be done in the unsaturated field of ear-worn health monitors.

%\begin{figure}[h]
%   \centering
%   \includegraphics[scale=0.5]{figs/TraditionalICU}
%  \caption{Traditional medical sign monitoring vs. The Ear-%Monitor}
%   \label{fig:TraditionalICU}
%\end{figure}